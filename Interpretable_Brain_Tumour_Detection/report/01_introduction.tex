\section{Introduction}
\label{sec:introduction}
Brain tumor is one of the leading death causing diseases. Several techniques are used to diagnose brain tumor such as computerized tomography (CT) scan, electroencephalogram (EEG), but magnetic resource imaging (MRI) is the most effective and widely used method. In an MRI, magnetic fields and radio waves are utilized to generate internal images of the organs within the body. Since MRI provides more detailed information on the internal organs, it is more effective than other techniques such as CT or EEG.

Given the complex nature of tumors such as abnormalities in their sizes and their location, it is very difficult to completely understand them. In the absence of a skillful doctor such as a professional neurosurgeon, it is quite challenging to analyze the MRI reports. A radiologist generates reports from MRI manually which can be prone to errors and is quite time-consuming. Therefore, an automated system can be very helpful in clinical assistance.

In this work, we exploit machine learning and deep learning algorithms and their capabilities to recognize complex data patterns, extract features automatically, and capture long range dependencies to detect tumor in given MRI and radiomics dataset. It is extremely important to accurately interpret and explain a model’s predictions. It helps to gain insight how a model can be improved and supports understanding of the process being modeled, thereby enhancing user trust. Therefore, we evaluate as well as interpret the performance of our ML and DL models and compare their interpretability and explainability.

