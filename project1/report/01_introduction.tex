\section{Introduction}
\label{sec:introduction}
An electrocardiogram (ECG) records the heart's rhythm through monitoring its electrical activity that occur during depolarization and repolarization of cardiac muscle during each heartbeat. 
%Specifically, electrodes placed on the body detect electrical changes that occur during depolarization and repolarization of cardiac muscle during each heartbeat. 
Cardiac abnormalities or cardiovascular diseases can cause changes to the ECG pattern and as such automatic classification and interpretation of these ECG signals may be essential for a timely diagnosis.

In this work, we exploit deep learning and its capabilities to recognize complex data patterns, extract features automatically, and capture long range dependencies for ECG interpretation using the PTB Diagnostic ECG Database and heart arrhythmia classification using the MIT-BIH Arrhythmia Database. We propose various deep-learning based models and evaluate their performance across these two tasks as well as compare them with traditional ML models. Transfer learning and ensemble methods are further explored to potentially increase performance.

%Classification and interpretation of these ECG signals are essential for timely diagnosis of cardiovascular diseases. Deep learning models can recognize complex data patterns, extract features automatically, and have a strong nonlinear fitting ability. This is very useful in identification and classification of unbalanced ECG datasets.

%We propose various models and apply them on the two datasets that have been provided, in order to establish which model provides the best results for classification of heart beats.
