\section{Introduction}
A randomized controlled trial (RCT) is a study where participants from an eligible group are assigned randomly to an experimental group or a control group. The study tests the extent to which specific, planned impacts are being achieved. RCTs provide the most compelling medical evidence. Since there are massive amounts of RCT data available, automatic sequential sentence classification of this data will be very useful in literature reviews. NLP can also be used for automating other interesting use cases such as summarizing text, information extraction such as the scientific claim of an abstract or information retrieval such as effect of a drug on a disease.\\
In this work, we exploit NLP with traditional machine learning and deep learning and its capabilities to recognize complex data patterns, extract features automatically, and capture long range dependencies for PubMed 200k RCT dataset. We propose various machine learning, deep learning and transformer based models and evaluate their performance on this task and compare them. 
\label{sec:introduction}
